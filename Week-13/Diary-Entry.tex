% Options for packages loaded elsewhere
\PassOptionsToPackage{unicode}{hyperref}
\PassOptionsToPackage{hyphens}{url}
%
\documentclass[
]{article}
\usepackage{amsmath,amssymb}
\usepackage{iftex}
\ifPDFTeX
  \usepackage[T1]{fontenc}
  \usepackage[utf8]{inputenc}
  \usepackage{textcomp} % provide euro and other symbols
\else % if luatex or xetex
  \usepackage{unicode-math} % this also loads fontspec
  \defaultfontfeatures{Scale=MatchLowercase}
  \defaultfontfeatures[\rmfamily]{Ligatures=TeX,Scale=1}
\fi
\usepackage{lmodern}
\ifPDFTeX\else
  % xetex/luatex font selection
\fi
% Use upquote if available, for straight quotes in verbatim environments
\IfFileExists{upquote.sty}{\usepackage{upquote}}{}
\IfFileExists{microtype.sty}{% use microtype if available
  \usepackage[]{microtype}
  \UseMicrotypeSet[protrusion]{basicmath} % disable protrusion for tt fonts
}{}
\makeatletter
\@ifundefined{KOMAClassName}{% if non-KOMA class
  \IfFileExists{parskip.sty}{%
    \usepackage{parskip}
  }{% else
    \setlength{\parindent}{0pt}
    \setlength{\parskip}{6pt plus 2pt minus 1pt}}
}{% if KOMA class
  \KOMAoptions{parskip=half}}
\makeatother
\usepackage{xcolor}
\usepackage[margin=1in]{geometry}
\usepackage{graphicx}
\makeatletter
\def\maxwidth{\ifdim\Gin@nat@width>\linewidth\linewidth\else\Gin@nat@width\fi}
\def\maxheight{\ifdim\Gin@nat@height>\textheight\textheight\else\Gin@nat@height\fi}
\makeatother
% Scale images if necessary, so that they will not overflow the page
% margins by default, and it is still possible to overwrite the defaults
% using explicit options in \includegraphics[width, height, ...]{}
\setkeys{Gin}{width=\maxwidth,height=\maxheight,keepaspectratio}
% Set default figure placement to htbp
\makeatletter
\def\fps@figure{htbp}
\makeatother
\setlength{\emergencystretch}{3em} % prevent overfull lines
\providecommand{\tightlist}{%
  \setlength{\itemsep}{0pt}\setlength{\parskip}{0pt}}
\setcounter{secnumdepth}{-\maxdimen} % remove section numbering
\ifLuaTeX
  \usepackage{selnolig}  % disable illegal ligatures
\fi
\IfFileExists{bookmark.sty}{\usepackage{bookmark}}{\usepackage{hyperref}}
\IfFileExists{xurl.sty}{\usepackage{xurl}}{} % add URL line breaks if available
\urlstyle{same}
\hypersetup{
  pdftitle={Diary Entry},
  pdfauthor={Loy Yee Keen},
  hidelinks,
  pdfcreator={LaTeX via pandoc}}

\title{Diary Entry}
\author{Loy Yee Keen}
\date{2023-11-16}

\begin{document}
\maketitle

\hypertarget{week-9}{%
\section{Week 9}\label{week-9}}

\hypertarget{what-is-the-topic-that-you-have-finalised-answer-in-1-or-2-sentences.}{%
\subsubsection{(1) What is the topic that you have finalised? (Answer in
1 or 2
sentences).}\label{what-is-the-topic-that-you-have-finalised-answer-in-1-or-2-sentences.}}

The topic that I have finalised is ``Names'', specifically ``Is there a
rise in the gender-neutral names given to babies born in America?''

\hypertarget{what-are-the-data-sources-that-you-have-curated-so-far-answer-1-or-2-sentences.}{%
\subsubsection{(2) What are the data sources that you have curated so
far? (Answer 1 or 2
sentences).}\label{what-are-the-data-sources-that-you-have-curated-so-far-answer-1-or-2-sentences.}}

I have curated a dataset consisting of a list of names given to babies
born in the US each year, the gender of the babies, and the count of
each name per gender. The data sources span from the years 1880 to 2022.
These data sources are extracted from the website of the United States
Social Security Administration, an independent agency of the U.S.
federal government.

\hypertarget{week-10}{%
\section{Week 10}\label{week-10}}

\hypertarget{what-is-the-question-that-you-are-going-to-answer-answer-one-sentence-that-ends-with-a-question-mark-that-could-act-like-the-title-of-your-data-story}{%
\subsubsection{(1) What is the question that you are going to answer?
(Answer: One sentence that ends with a question mark that could act like
the title of your data
story),}\label{what-is-the-question-that-you-are-going-to-answer-answer-one-sentence-that-ends-with-a-question-mark-that-could-act-like-the-title-of-your-data-story}}

Is there a rise in the gender-neutral names given to babies born in
America?

\hypertarget{why-is-this-an-important-question-answer-3-sentences-each-of-which-has-some-evidence-e.g.-according-to-the-united-nations-to-justify-why-the-question-you-have-chosen-is-important}{%
\subsubsection{\texorpdfstring{(2) Why is this an important question?
(Answer: 3 sentences, each of which has some evidence, e.g., ``According
to the United Nations\ldots{}'' to justify why the question you have
chosen is
important)}{(2) Why is this an important question? (Answer: 3 sentences, each of which has some evidence, e.g., ``According to the United Nations\ldots'' to justify why the question you have chosen is important)}}\label{why-is-this-an-important-question-answer-3-sentences-each-of-which-has-some-evidence-e.g.-according-to-the-united-nations-to-justify-why-the-question-you-have-chosen-is-important}}

The Council of Europe underscores the role of gender in shaping power
dynamics and opportunities in society. The popularity of gender-neutral
names reflects a broader shift toward inclusivity and the challenge of
traditional gender roles.

\emph{Source:
\url{https://www.coe.int/en/web/gender-matters/exploring-gender-and-gender-identity\#:~:text=Gender\%20is\%20of\%20key\%20importance,equality\%20and\%20freedom\%20from\%20discrimination}}

~

Additonally, gender-neutral names empower girls and women by challenging
gender stereotypes. According to a New York Times article, some parents
opt for these names to counter biases and promote strength for their
daughters.

\emph{Source:
\url{https://nypost.com/2018/03/21/why-gender-neutral-baby-names-are-on-the-rise/}}

~

This trend aligns with the United Nations' Sustainable Development Goal
5, which aims to achieve gender equality and empower women and girls.

\emph{Source: \url{https://sdgs.un.org/goals/goal5}}

\hypertarget{which-rows-and-columns-of-the-dataset-will-be-used-to-answer-this-question-answer-actual-names-of-the-variables-in-the-dataset-that-you-plan-to-use.}{%
\subsubsection{(3) Which rows and columns of the dataset will be used to
answer this question? (Answer: Actual names of the variables in the
dataset that you plan to
use).}\label{which-rows-and-columns-of-the-dataset-will-be-used-to-answer-this-question-answer-actual-names-of-the-variables-in-the-dataset-that-you-plan-to-use.}}

I will use multiple datasets to answer the chosen question. All the
datasets have the same format; each dataset represents a specific year
spanning from 1882 to 2022.

In each dataset, there are 3 columns, corresponding to the name of the
baby, the sex of the baby and the count of babies with that name (The
original dataset did not define the names of the variables, so I will
redefine the variables as Name, Sex and Count).

The number of rows, each corresponding to the observation for each name,
differ for every year.

Every row and column of the datasets will be used to answer my chosen
question as all of them are relevant in comparing the shifts in naming
trends over the years.

I will use rbind to combine the datasets into a single dataset.

\hypertarget{challenges-and-errors-that-you-faced-and-how-you-overcame-them.}{%
\subsubsection{(4) Challenges and errors that you faced and how you
overcame
them.}\label{challenges-and-errors-that-you-faced-and-how-you-overcame-them.}}

I encountered difficulties when I read the files using read\_csv because
the datasets did not have column names. Consequently, the output
assigned the first value in each cell of the respective columns as the
column names. This approach was erroneous, as the data in the first row
represented observations, not variables. To resolve this issue, I tried
to look up the answer in the textbook reading
(\url{https://r4ds.hadley.nz/data-import}) provided in the Lecture 9
slides.

\hypertarget{week-11}{%
\section{Week 11}\label{week-11}}

\hypertarget{list-the-visualisations-that-you-are-going-to-use-in-your-project}{%
\subsubsection{(1) List the visualisations that you are going to use in
your
project}\label{list-the-visualisations-that-you-are-going-to-use-in-your-project}}

\begin{itemize}
\tightlist
\item
  I. Barplot of proportion of US babies with gender-neutral names by
  year:

  \begin{itemize}
  \tightlist
  \item
    Variables: y axis = Proportion of gender-neutral names; x-axis =
    Year.
  \item
    Purpose: Investigate the trend over time, determining whether there
    is an increase in the use of gender-neutral names for babies in the
    US.
  \end{itemize}
\end{itemize}

\begin{enumerate}
\def\labelenumi{\Roman{enumi}.}
\setcounter{enumi}{1}
\tightlist
\item
  Table of gender-neutral names in 2022:
\end{enumerate}

\begin{itemize}
\tightlist
\item
  Columns: Gender-neutral names in 2022, count of male babies, count of
  female babies, total count and the proportion overlap between the
  counts of both sexes.
\item
  Purpose: Provides a more detailed visualisation of the specific names
  of which the trends would be plotted in III.
\end{itemize}

\begin{enumerate}
\def\labelenumi{\Roman{enumi}.}
\setcounter{enumi}{2}
\tightlist
\item
  5 line plots of the trend of gender-neutral name over time, grouped by
  gender:
\end{enumerate}

\begin{itemize}
\tightlist
\item
  Variables: x-axis: Proportion of babies with names that at least 90\%
  male-female overlap, y-axis: Year
\item
  Purpose: Analyse the trends of these names, understanding whether
  these names have been consistently gender-neutral or have transitioned
  from a more gender-specific association over time. This could possibly
  answer the question of why there is a rise/fall in gender-neutral
  names over time.
\end{itemize}

\hypertarget{how-do-you-plan-to-make-it-interactive}{%
\subsubsection{(2) How do you plan to make it
interactive?}\label{how-do-you-plan-to-make-it-interactive}}

\begin{itemize}
\tightlist
\item
  I. Features:

  \begin{itemize}
  \tightlist
  \item
    A slider for adjusting the number of bars in the barplot.
  \item
    Radio buttons displaying the proportion of male and female babies
    with gender-neutral names.
  \end{itemize}
\end{itemize}

\begin{enumerate}
\def\labelenumi{\Roman{enumi}.}
\setcounter{enumi}{1}
\tightlist
\item
  Features:
\end{enumerate}

\begin{itemize}
\tightlist
\item
  A button to highlight values in the table which have more than 90\%
  male-female proportion overlap.
\item
  numbericInput function to choose the number of rows to display in the
  table.
\end{itemize}

\begin{enumerate}
\def\labelenumi{\Roman{enumi}.}
\setcounter{enumi}{2}
\tightlist
\item
  Features:
\end{enumerate}

\begin{itemize}
\tightlist
\item
  SelectInput function to choose a name and display the corresponding
  plot.
\item
  Forward (and backward) navigation buttons to show the plot for the
  next 50 years.
\item
  A card providing an explanation of the plot, updating itself when the
  navigation buttons are pressed.
\item
  Tab panel that displays all the 5 plots in a single view (using facet
  wrap).
\end{itemize}

\hypertarget{what-concepts-incorporated-in-your-project-were-taught-in-the-course-and-which-ones-were-self-learnt}{%
\subsubsection{(3) What concepts incorporated in your project were
taught in the course and which ones were
self-learnt?}\label{what-concepts-incorporated-in-your-project-were-taught-in-the-course-and-which-ones-were-self-learnt}}

\begin{verbatim}
## Warning: package 'readxl' was built under R version 4.2.3
\end{verbatim}

\begin{verbatim}
## Warning: package 'dplyr' was built under R version 4.2.3
\end{verbatim}

\begin{verbatim}
## 
## Attaching package: 'dplyr'
\end{verbatim}

\begin{verbatim}
## The following objects are masked from 'package:stats':
## 
##     filter, lag
\end{verbatim}

\begin{verbatim}
## The following objects are masked from 'package:base':
## 
##     intersect, setdiff, setequal, union
\end{verbatim}

\begin{verbatim}
## # A tibble: 61 x 2
##    Topic                                                    Week 
##    <chr>                                                    <chr>
##  1 library                                                  2    
##  2 pull                                                     2    
##  3 logical operators (==, &, <, >)                          2 & 4
##  4 ggplot                                                   2 & 7
##  5 as.integer                                               3    
##  6 as.character                                             3    
##  7 vector('list', length=)                                  3    
##  8 list[['']]                                               3    
##  9 c()                                                      3    
## 10 read_csv                                                 3    
## 11 $                                                        3    
## 12 filter                                                   4    
## 13 select                                                   4    
## 14 mutate                                                   4    
## 15 arrange(desc())                                          4    
## 16 seq(from=, to=, by=)                                     4    
## 17 slice                                                    4    
## 18 : eg 1:5                                                 4    
## 19 ifelse                                                   4    
## 20 functions                                                5    
## 21 paste0                                                   5    
## 22 for loop                                                 6    
## 23 aes(group=)                                              7    
## 24 aes(colour=)                                             7    
## 25 geom_col(fill=)                                          7    
## 26 geom_col(alpha=)                                         7    
## 27 guides()                                                 7    
## 28 facet_wrap(~)                                            8    
## 29 shiny                                                    8    
## 30 sliderInput                                              8    
## 31 unlist                                                   <NA> 
## 32 abs                                                      <NA> 
## 33 scales = 'free_y'                                        <NA> 
## 34 geom_line                                                <NA> 
## 35 read_csv(col_names=)                                     <NA> 
## 36 merge                                                    <NA> 
## 37 data.frame                                               <NA> 
## 38 rbind                                                    <NA> 
## 39 bind_rows                                                <NA> 
## 40 do.call                                                  <NA> 
## 41 css                                                      <NA> 
## 42 is.null                                                  <NA> 
## 43 reactiveVal                                              <NA> 
## 44 observeEvent                                             <NA> 
## 45 plotly                                                   <NA> 
## 46 DT package                                               <NA> 
## 47 scale_colour_manual                                      <NA> 
## 48 guide_legend(title = NULL)                               <NA> 
## 49 geom_col                                                 <NA> 
## 50 as.double(digits=2)                                      <NA> 
## 51 scale_x_continuous/scale_y_continuous(limits=, breaks=) <NA> 
## 52 radioButtons                                             <NA> 
## 53 numericInput                                             <NA> 
## # i 8 more rows
\end{verbatim}

\textbf{Explanations for some of the functions I used are as follows:}

\textbf{filter():} filter Sex == ``M'' and filter Sex =``F'' from the
original dataset and store in a new variable each

\textbf{mutate():} create new columns in the dataframe called
\texttt{Count\ of\ Female\ Babies}, \texttt{Count\ of\ Male\ Babies},
\texttt{Total\ Count} and
\texttt{Proportion\ Overlap\ Between\ M\ \&\ F}

\textbf{arrange(desc()):} sort dataframe in descending order of
\texttt{Total\ Count}

\textbf{seq(from=1882, to=2022, by=10):} set the breaks of the bar plot
to be in intervals of 10 from 1882 to 2022

\textbf{ifelse:} if radio buttons are clicked, display its corresponding
plot in the output of distPlot1 highlight the names in the shiny output
table if \texttt{Proportion\ Overlap\ Between\ M\ \&\ F} is more than
0.9 plot the lines for the next 50 years and update the text if the
forward button is pressed, reverse if backward button is pressed print
the text corresponding to a particular name if that name is selected in
the selectInput function on shiny

\textbf{functions} function to filter male and female names function to
merge the dataframes for male and female into one dataframe called
gender-neutral names (using \textbf{merge()} function) function to read
dataset for each year (see paste0 function below)

\textbf{paste0():} concatenate the strings: ``yob'', year \& ``.txt'' as
they have no separators read\_year\_data \textless- function(year) \{
data \textless- read\_csv(paste0(``yob'', year, ``.txt''), col\_names =
c(``Name'', ``Sex'', ``Count'')) \}

\textbf{for loop:} execute functions across multiple names and multiple
years from 1882 to 2022

\textbf{geom\_col():} bar plot of the proportion of gender-neutral names
for each year and another bar plot of the proportion of names occurring
less than 50 times for each year (geom\_col is used as y variable is
proportion not count)

\textbf{geom\_col(fill=)}: fill the bars with different colors based on
the male and female proportions

\textbf{geom\_col(alpha=)}: to make the plot translucent so the
overlapping colours for male and female proportions can be compared more
clearly

\textbf{ggplot(data)+aes(group=Sex, color=Sex)+geom\_line()}: plot a
line plot of proportion of babies with a particular gender-neutral name,
with 2 separate lines of different colours for male and female babies.

\textbf{facet\_wrap(\textasciitilde{} Name, scales = ``free\_y'')}: plot
each name in a separate plot and show all the plots at the same time and
allow the scales of the y-axis to vary between facets based on the data
for each `Name' otherwise some of the y-axes will be so large that the
plot cannot be seen clearly

\textbf{reactiveVal():} store the selected name in selectInput

\textbf{observeEvent():} observe changes in this stored value (the
selected name), and updates the plot/text whenever the user selects a
new name or presses a button

\textbf{switch} In this code chunk, selected\_data \textless-
switch(input\(proportion_per_sex,  "Male" = filtered_proportion_m_df,  "Female" = filtered_proportion_f_df  ), the value of input\)proportion\_per\_sex
determines which alternative is selected. If it's ``Male,''
selected\_data will be assigned the value of
filtered\_proportion\_m\_df; if it's ``Female,'' it will be assigned the
value of filtered\_proportion\_f\_df

\textbf{rbind}: combined\_name\_trend\_df \textless- do.call(rbind,
name\_trend\_list) Combine name\_trend\_list for all the years into a
single dataframe, combined\_name\_trend\_df

\hypertarget{include-the-challenges-and-errors-that-you-faced-and-how-you-overcame-them.}{%
\subsubsection{(4) Include the challenges and errors that you faced and
how you overcame
them.}\label{include-the-challenges-and-errors-that-you-faced-and-how-you-overcame-them.}}

The first error I faced was when I used a ``for loop'' that loops
through each dataframe for the years 1882 to 2022, and merges the names
that are given to both male and female babies. However, when I tried to
access the output outside the loop, I was returned ``Error: object
`gender\_neutral\_names' not found''.

To overcome this error, I went back to lecture 6 about ``for loops'' and
recalled that we had to pre-allocate space to store the output. I then
corrected the code by adding ``gender\_neutral\_names \textless-
vector(''list'', length =1882:2022)'' before the loop.

\begin{center}\rule{0.5\linewidth}{0.5pt}\end{center}

The next error I faced was when I tried to access a year from the
gender\_neutral\_names list using gender\_neutral\_names(``2022''), but
was returned with Error in gender\_neutral\_names(``2022'') : could not
find function ``gender\_neutral\_names''.

I then referred to lecture 3 and realised that we need to access columns
in a list using gender\_neutral\_names{[}{[}``2022''{]}{]} instead.

\begin{center}\rule{0.5\linewidth}{0.5pt}\end{center}

Another error I faced was when I tried to arrange the data in a column
named ``Total Count''. However, when I printed the output, the data was
not arranged.

To overcome this challenge, I went online to search for how to reference
to column names that include spaces and found out that we need to use
backticks around the column name.

\#Week-12 Challenges

For my first visualisation, I tried to plot a barplot of the proportion
of gender-neutral names given to babies per year using

ggplot(gender\_neutral\_names) + aes(x = year, y = proportion) +
geom\_bar()

However, I was returned with ``Error occurred in the 1st layer. Caused
by error in \texttt{setup\_params()}:! \texttt{stat\_count()} must only
have an x or y aesthetic''.

To overcome this error and plot the barplot successfully, I copied this
error into Google. The first result came from stackoverflow which said
that I need to either use geom\_col() or geom\_bar(stat=''identity'').
This is because the default for geom\_bar() is (stat=count) which counts
the aggregate number of rows for each x value (year), but I am instead
providing the y-values (proportion) for the barplot.

I also included radio buttons (none/male/female/both) that fill this
gender-neutral barplot according to the proportion of male/female/both
that are given gender-neutral names.

I initially tried to do this by binding these 3 dataframes together
using rbind, and then write a code using ifelse, something like

if(input\(proportion_per_sex=female) { plot<-ggplot(total_proportion)+ aes(x=year, y=proportion, fill=female_proportion) } else if(input\)proportion\_per\_sex=male)
\{ plot\textless-ggplot(total\_proportion)+ aes(x=year, y=proportion,
fill=male\_proportion) \} else \{plot\textless-total\_proportion)
+aes(x=year, y=proportion) \}

but this doesn't work since female\_proportion and male\_proportion are
separate columns in the combined dataframe and fill works by colouring
binary factor variables within the same column\\
Instead, I googled if it was possible to layer multiple ggplots in the
same plot using eg

ggplot() + geom\_col(data = total\_proportion) + aes(x = year, y =
proportion) + geom\_col(data = male\_proportion) + aes(x = year, y =
proportion) + geom\_col(data = female\_proportion) + aes(x = year, y =
proportion)

Apparently it was possible, so I used this method instead.

For my second visualisation, I had 2 navigation buttons in the side
panel that display the line plot when pressed. For example, the original
plot (without pressing any buttons) displays an empty plot with x-axis
from 1882 to 2022. When the forward button is pressed once, it displays
the line from 1882 to 1932 (50 years), then when it is pressed again, it
displays from 1882 to 1982 (next 50 years). To accumulate the years, I
tried to use \textless\textless- which I learnt during Week 5's
tutorial. My original code was interval\_end \textless\textless-
interval\_end + 50. However, the plot is empty when I rendered it.

For this visualisation to work effectively, I created two functions, one
to store the interval end, and one to store the cumulative end
(interval\_end \textless- reactiveVal(1882) and cumulative\_end
\textless- reactiveVal(1882)), and used yet another function
new\_cumulative\_end \textless- cumulative\_end() + 50 to accumulate the
years instead.

I also faced another challenge for this plot. Even if a name only has
records starting from a specific year that is not 1882 (for instance,
1960), the plot does not start the line from 1960 but instead extends
the line back 1882, which would otherwise erroneously imply the name's
existence before it was actually established. Also, the colour aesthetic
for the plot gets mixed up. For instance, if a name only has records for
male babies in the 1960s, the plot displays a pink line to represent
this trend. However, when there are records for both sexes in the later
years, say 1980, the plot displays 2 lines, but now the male line is
blue while the female line is pink. Hence, I included a code chunk to
set the proportion to 0 when the year is 1882.

\#Final Write-up

\#\#Theme of Datastory and its Importance Focusing on the question of
whether there is a growing prevalence of gender-neutral names in the
U.S., this datastory explores names as indicators of evolving societal
attitudes. If there is indeed a growing prevalence, it may signify a
society embracing gender fluidity ((Kihm, n.d., as cited in Mustafa,
2023) and challenging traditional gender stereotypes (Mahdawi, 2016),
holding implications for fostering inclusivity and advocating an
egalitarian culture. By unravelling the trends in baby naming, we gain
insights into the broader narrative of societal progress.

\#\#Curated Data Sources To answer this question, 141 datasets spanning
from 1882 to 2022 were compiled. These datasets not only cover a vast
timeline but are also sourced from the official records of the Social
Security Administration, a federal agency of the U.S. government. Thus,
they serve as a comprehensive and reliable source for analysing the
historical trends of names given to babies born in the U.S.

To streamline the analysis, the following criteria were established to
define gender-neutral names The difference in counts between both sexes
is less than 300 The count for either sex is less than double that of
the opposite sex

\#Implementation of the Project

\#\#First Plot In exploring the evolving prevalence of gender-neutral
names and aiming to visually represent the proportion of babies given
gender-neutral names over the years, I opted for a barplot due to its
clarity in illustrating the distribution of data. However, given that
proportion is represented on the y-axis, I used geom\_col() instead of
geom\_bar(). Unlike geom\_bar(), which counts the frequency of
occurrences on the x-axis and plots them on the y-axis, geom\_col()
allows for the direct plotting of the pre-calculated values on the
y-axis.

To elevate the user experience, I introduced a slider that allows users
to adjust the number of bars displayed. Alongside this, I integrated
radio buttons, utilising the switch function for seamless toggling
between visualisations of proportions for male or female babies with
gender-neutral names. These visualisations overlay the original plot,
allowing users to compare the distribution of gender-neutral names among
male and female babies.

To further enrich the visual representation, I introduced an additional
radio button labeled ``Both'', allowing users to concurrently observe
the overlay of male and female proportions of babies with gender-neutral
names on the original plot. However, as the switch function is designed
for toggling between distinct visualisations, I drew upon the concepts
introduced in Lectures 2 and 7 that ggplots are composed of layers, and
self-experimented with the feasibility of using multiple + geom\_col()
functions to layer the plots of both male and female proportions.

\#\#\#Insights from First Plot

The barplot depicted a significant rise in the popularity of
gender-neutral names over the last 140 years. Interestingly, the
proportion of gender-neutral names in the late 1800s was quite high, but
the trend declined, reaching its lowest point around the 1960s. It was
not until the late 1900s that the trend experienced a resurgence and has
since increased exponentially, more than doubling since the 1800s. This
provides testament to the shift in preference for gender-neutral names
over the years, which is perhaps reflective of the increased acceptance
of gender fluidity, prompting parents to seek names that transcend
traditional gender boundaries.

The plot also reveals a shift in the gender distribution of these names.
Initially, during the late 1800s to early 1900s, gender-neutral names
were more prevalent among male babies. However, the proportion has since
become more spread out across both sexes, often more common among female
babies, especially during the mid-late 1900s. This shift suggests that
parents are increasingly open to the idea of giving their daughters
gender-neutral names, challenging the traditional stereotypes that
dictate how names should align with gender norms. This shift perhaps
reflects a broader societal advancement toward a culture that
prioritises gender equality.

\#\#Second Plot To delve deeper into the historical associations of
names currently considered gender-neutral and understand their
evolution---whether they were consistently popular among both sexes or
if there was a historical bias towards a particular sex, potentially
more commonly associated with males---I studied the trends of the top 10
most popular gender-neutral names of 2022. However, to ensure these
names are popular across both sexes, I only included those with a
proportion overlap of more than 90\% in the counts of both male and
female babies. There were 5 such names out of the 10.

To effectively showcase these trends, I opted for a line plot for each
of the names, grouped and coloured by sex to facilitate a clearer
comparison between the sexes. A dropdown menu allows users to select a
name, instantly displaying its corresponding plot. Each plot is
accompanied by text explanations within a div element---a self-learned
concept. I also incorporated navigation buttons that, when clicked,
reveal subsequent 50-year intervals of the plot. The reactiveVal()
function, which was also learned independently, accumulated the years
when these buttons were clicked. Additionally, I introduced another tab
displaying all the plots combined in a single view to facilitate a more
effective comparison of the trends across the 5 names.

However, given the extensive time span of 140 years on the x-axis, there
were certain parts that were hard to see, hence I explored options for
zoom functionality. My research led me to discover that Plotly offers a
solution for this purpose.

\#\#\#Insights from Second Plot

The plots provide some support for the hypothesis that names currently
considered gender-neutral used to be more commonly associated among
males. 3 out of the 5 names (Blake, Charlie, and Finley) were originally
male-associated but have seen a recent surge in popularity among female
babies. Blake and Charlie used to be highly popular among male babies,
but this popularity has fallen drastically over the years. As of 2022,
more female babies are named Charlie than male babies, and the same
trend is also observed for Finley. On the contrary, none of the 5 names
were predominantly female-associated in the past.

This trend could potentially be attributed to parents wanting to
challenge traditional gender stereotypes, as proposed earlier. This
could be due to potential advantages in the workplace for females with
less explicitly feminine names, as they are seen as more competent
(Mahdawi, 2016). In fact, 2 out of the 5 names mean ``warrior'',
projecting a sense of capability.

\#\#Conclusion:

The rise of gender-neutral names in the U.S. mirrors a broader societal
shift towards inclusivity, challenging traditional gender boundaries. In
exploring these trends, this datastory not only uncovered historical
trends but also provided insights into the evolving associations of
specific names. As society embraces more egalitarian values, the choice
of gender-neutral names stands as a testament to this evolving cultural
landscape.

\#\#References

Mustafa, T. (2023, September). The gender neutral baby names taking the
UK by storm. Metro.
\url{https://metro.co.uk/2023/09/27/unisex-baby-names-rise-as-parents-want-gender-neutral-options-19544405/\#}:\textasciitilde:text=She\%20says\%3A\%20'Gender\%20neutral\%20names,modern\%20choices\%20with\%20contemporary\%20style.

Mahdawi, A. (2016, September). Initial impressions: will hiding my name
force gender equality in the workplace?. The Guardian.
\url{https://www.theguardian.com/lifeandstyle/2016/sep/29/gender-neutral-name-equality-work}

Social Security Administration. (n.d.). National Data. (1882-2022).
\url{https://www.ssa.gov/oact/babynames/limits.html}

Word Count: 1191

\end{document}
